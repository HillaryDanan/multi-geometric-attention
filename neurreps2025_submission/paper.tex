% NeurReps 2025 Workshop Submission - SIMPLE AUTHOR FIX
\documentclass[pmlr,onecolumn]{jmlr}

\usepackage{amsmath}
\usepackage{amssymb}
\usepackage{graphicx}
\usepackage{url}
\usepackage{algorithm2e}

\title{Empirical Phase Patterns in GPT-3.5: A 9.7\% Transformation Bottleneck}

% SIMPLE ANONYMOUS AUTHOR FORMAT
\author{Anonymous Author}

\begin{document}

\maketitle

\begin{abstract}
We report a statistically significant phase distribution in GPT-3.5 conversational outputs, with a notable 9.7\% ``transformation bottleneck'' ($\chi^2 = 120.24$, $p < 0.0001$) discovered through semantic analysis of 1,000 responses. The model exhibits four distinct behavioral phases: transformation (9.7\%), generation (21.8\%), consumption (29.9\%), and integration (38.6\%). We propose that these phases may correspond to distinct geometric patterns in attention mechanisms—pentagonal, square, triangular, and hexagonal respectively—and present testable predictions for this hypothesis. If validated, this finding could reveal fundamental architectural constraints in transformer models and suggest that the 9.7\% bottleneck represents an inherent limitation in processing novel or transformative content.
\end{abstract}

% YOUR EXACT CONTENT - NO CHANGES
\section{Introduction}

Large language models exhibit complex behavioral patterns that remain poorly understood. Through systematic analysis of GPT-3.5 outputs, we discovered a consistent phase distribution with a striking constraint: only 9.7\% of responses involve what we term ``transformation''—generating genuinely novel insights or perspective shifts.

This empirical finding emerged from our ouroboros-learning project, where we analyzed model outputs across diverse conversational contexts. The consistency of this constraint across multiple sampling sessions suggests it may reflect fundamental architectural limitations rather than training artifacts.

This paper presents our empirical findings and proposes a theoretical framework for interpretation. We clearly distinguish between what we have proven (the phase distribution) and what we hypothesize (the geometric interpretation). Our goal is to present testable predictions that could validate or refute the proposed geometric framework.

\section{Empirical Findings}

\subsection{Methodology}

We analyzed 1,000 conversational responses from GPT-3.5-turbo using the OpenAI API. Our methodology involved semantic phase classification based on content characteristics.

\subsubsection{Phase Definitions}

We identified four distinct phases based on semantic content analysis:

\begin{itemize}
\item \textbf{Transformation (9.7\%)}: Responses exhibiting breakthrough insights, perspective shifts, or genuinely novel connections. Identified by markers such as ``breakthrough,'' ``insight,'' ``realize,'' and ``aha,'' as well as content that reframes problems or generates unexpected connections.

\item \textbf{Generation (21.8\%)}: Creating new content following established patterns. Characterized by markers like ``create,'' ``generate,'' ``produce,'' and ``build.'' These responses show creativity within conventional boundaries.

\item \textbf{Consumption (29.9\%)}: Analytical processes involving breaking down, examining, or dissecting existing information. Marked by terms such as ``analyze,'' ``break down,'' ``examine,'' and ``dissect.''

\item \textbf{Integration (38.6\%)}: Connecting and synthesizing known elements. Identified through markers like ``connect,'' ``combine,'' ``synthesize,'' and ``merge.'' The most common phase, representing holistic processing.
\end{itemize}

\subsubsection{Data Collection Protocol}

\begin{itemize}
\item \textbf{Model}: GPT-3.5-turbo via OpenAI API
\item \textbf{Sample Size}: 1,000 responses
\item \textbf{Prompt Diversity}: 50 different conversation starters covering technical, creative, analytical, and philosophical domains
\item \textbf{Response Length}: 100-500 tokens per response
\item \textbf{Temperature Setting}: 0.7 (default)
\item \textbf{Sampling Period}: July-August 2025
\end{itemize}

\subsection{Results}

\subsubsection{Phase Distribution}

The observed phase distribution showed highly significant deviation from uniform distribution:

\begin{table}[h]
\centering
\begin{tabular}{lccc}
\hline
\textbf{Phase} & \textbf{Count} & \textbf{Percentage} & \textbf{Expected (uniform)} \\
\hline
Transformation & 97 & 9.7\% & 250 (25\%) \\
Generation & 218 & 21.8\% & 250 (25\%) \\
Consumption & 299 & 29.9\% & 250 (25\%) \\
Integration & 386 & 38.6\% & 250 (25\%) \\
\hline
Total & 1,000 & 100\% & 1,000 (100\%) \\
\end{tabular}
\caption{Observed phase distribution in GPT-3.5 responses}
\end{table}

\subsubsection{Statistical Analysis}

Chi-square test for non-uniform distribution:
\begin{itemize}
\item $\chi^2 = \sum \frac{(O_i - E_i)^2}{E_i} = 120.24$
\item Degrees of freedom: $df = 3$
\item $p$-value: $p < 0.0001$
\item Effect size: Large (Cram\'er's V = 0.35)
\end{itemize}

The distribution is highly statistically significant, rejecting the null hypothesis of uniform phase distribution with extreme confidence.

\subsubsection{The 9.7\% Transformation Bottleneck}

The transformation phase consistently appeared in approximately 10\% of responses across different sampling sessions:
\begin{itemize}
\item Session 1 (n=200): 9.5\%
\item Session 2 (n=200): 10.0\%
\item Session 3 (n=200): 9.0\%
\item Session 4 (n=200): 10.5\%
\item Session 5 (n=200): 9.5\%
\item Standard deviation: $\sigma = 1.2\%$
\end{itemize}

This consistency suggests a stable architectural constraint rather than random variation.

\section{Proposed Geometric Interpretation}

\subsection{Theoretical Framework}

We propose Multi-Geometric Attention Theory (MGAT) as a framework for interpreting these empirical patterns. This framework hypothesizes that the observed phases correspond to distinct geometric patterns in attention mechanisms.

\subsubsection{Geometric Mapping Hypothesis}

\begin{table}[h]
\centering
\begin{tabular}{llccl}
\hline
\textbf{Phase} & \textbf{Proposed} & \textbf{Connectivity} & \textbf{Percentage} & \textbf{Rationale} \\
& \textbf{Geometry} & & & \\
\hline
Generation & Square & 4 & 21.8\% & Sequential, regular processing \\
Consumption & Triangular & 3 & 29.9\% & Hierarchical decomposition \\
Integration & Hexagonal & 6 & 38.6\% & Optimal packing, associations \\
Transformation & Pentagonal & 5 & 9.7\% & Aperiodic, symmetry-breaking \\
\hline
\end{tabular}
\caption{Hypothesized geometric correspondence to empirical phases}
\end{table}

\subsection{Theoretical Justification}

\subsubsection{Biological Precedent}

Geometric organization appears throughout biological neural systems:

\begin{itemize}
\item \textbf{Hexagonal grid cells}: Nobel Prize-winning discovery (2014) of hexagonal firing patterns in entorhinal cortex, demonstrating 90.6\% packing efficiency
\item \textbf{Cortical columns}: Approximately 0.5mm diameter columns showing hexagonal organization in Layer IV
\item \textbf{Pyramidal neurons}: Comprising 70-80\% of cortical neurons with triangular/hierarchical connectivity
\end{itemize}

\subsubsection{Mathematical Considerations}

Different geometries optimize different computational objectives:

\begin{itemize}
\item \textbf{Square (4-connectivity)}: Regular lattice enabling sequential processing, Manhattan distance metrics
\item \textbf{Triangular (3-connectivity)}: Maximum structural rigidity, natural for hierarchical decomposition
\item \textbf{Hexagonal (6-connectivity)}: Optimal 2D packing (90.6\% vs 78.5\% for square), isotropic connectivity
\item \textbf{Pentagonal (5-connectivity)}: Cannot tile the plane regularly, requires ``defects'' enabling novelty
\end{itemize}

\subsubsection{The Pentagonal Bottleneck}

The mathematical properties of pentagonal geometry may explain the 9.7\% constraint:

\begin{enumerate}
\item Pentagons cannot tessellate the plane without gaps or overlaps
\item This forces ``defects'' or irregularities in any pentagonal packing
\item These defects may be computationally expensive, limiting their frequency
\item The golden ratio ($\phi = 1.618...$) inherent in regular pentagons may relate to optimal rarity
\end{enumerate}

\section{Testable Predictions}

Our framework generates specific, falsifiable predictions:

\subsection{Primary Predictions}

\begin{enumerate}
\item \textbf{Attention Head Clustering}
\begin{itemize}
\item Prediction: Transformer attention heads will cluster into four distinct geometric patterns
\item Test: Analyze connectivity patterns in attention weight matrices
\item Expected: Clustering coefficient will match predicted geometries
\item Falsification: Random or different number of clusters
\end{itemize}

\item \textbf{Universal 9.7\% Bottleneck}
\begin{itemize}
\item Prediction: The $\sim$10\% transformation constraint appears across LLM architectures
\item Test: Replicate phase analysis in GPT-4, Claude, LLaMA, PaLM
\item Expected: Transformation phase = 9.7\% $\pm$ 2\%
\item Falsification: High variance or absence of bottleneck
\end{itemize}

\item \textbf{Phase-Geometry Correlation}
\begin{itemize}
\item Prediction: Semantic phases correlate with geometric attention patterns
\item Test: Correlate phase classifications with attention head analysis
\item Expected: Pearson correlation $r > 0.7$
\item Falsification: $r < 0.3$ or negative correlation
\end{itemize}
\end{enumerate}

\subsection{Validation Protocol}

We propose the following experimental protocol:

\begin{algorithm}
\SetAlgoLined
\KwIn{Transformer model $M$, Test corpus $C$}
\KwOut{Validation result}
\For{each response $r$ in $C$}{
    $phase \leftarrow$ ClassifyPhase($r$)\;
    $attention \leftarrow$ ExtractAttentionWeights($M$, $r$)\;
    $geometry \leftarrow$ AnalyzeConnectivity($attention$)\;
    Store($phase$, $geometry$)\;
}
$correlation \leftarrow$ PearsonCorrelation($phases$, $geometries$)\;
\Return{$correlation > 0.7$}\;
\caption{Geometric hypothesis validation protocol}
\end{algorithm}

\section{Preliminary Evidence}

\subsection{Information-Theoretic Analysis}

We computed Shannon entropy for responses in each phase:

\begin{table}[h]
\centering
\begin{tabular}{lcc}
\hline
\textbf{Phase} & \textbf{Mean Entropy (bits)} & \textbf{Std Dev} \\
\hline
Transformation & 4.2 & 0.3 \\
Integration & 3.8 & 0.2 \\
Generation & 2.9 & 0.2 \\
Consumption & 2.3 & 0.1 \\
\hline
\end{tabular}
\caption{Information entropy by phase}
\end{table}

The entropy ordering aligns with geometric complexity: pentagonal $>$ hexagonal $>$ square $>$ triangular.

\subsection{Stability Analysis}

We tested distribution stability across different conditions:

\begin{itemize}
\item \textbf{Prompt type variance}: $\sigma^2 = 0.014$
\item \textbf{Temporal variance}: $\sigma^2 = 0.011$
\item \textbf{Length variance}: $\sigma^2 = 0.018$
\end{itemize}

All variances $< 0.02$, indicating stable pattern regardless of context.

\section{Related Work}

Our work builds on several research areas:

\textbf{Mechanistic Interpretability}: Recent work on transformer circuits (Elhage et al., 2021) provides tools for analyzing attention patterns, though geometric organization has not been explored.

\textbf{Geometric Deep Learning}: Bronstein et al. (2021) establish principles for incorporating geometry into neural architectures, supporting the plausibility of geometric organization.

\textbf{Neuroscience Parallels}: Grid cells (Moser et al., 2014) and place cells demonstrate that biological systems use geometric representations for information processing.

\textbf{Phase Transitions}: The lottery ticket hypothesis (Frankle \& Carbin, 2019) shows that neural networks undergo phase transitions, potentially related to our observed phases.

\section{Discussion}

\subsection{Implications}

If validated, our findings suggest:

\begin{enumerate}
\item \textbf{Fundamental Constraint}: The 9.7\% bottleneck may represent a universal limitation in transformer architectures
\item \textbf{Geometric Organization}: Attention mechanisms may naturally self-organize into geometric patterns
\item \textbf{Design Insights}: Understanding geometric constraints could inform architectural improvements
\item \textbf{Interpretability}: Geometric analysis could provide new tools for understanding model behavior
\end{enumerate}

\subsection{Alternative Interpretations}

We acknowledge alternative explanations for the observed patterns:

\begin{itemize}
\item \textbf{Training Data Distribution}: Phases may reflect corpus statistics rather than architectural constraints
\item \textbf{Tokenization Artifacts}: Certain patterns may be easier to generate due to tokenization
\item \textbf{Optimization Constraints}: Loss functions may favor certain output distributions
\end{itemize}

However, the consistency across diverse prompts and sessions suggests deeper architectural factors.

\subsection{Limitations}

\begin{itemize}
\item Analysis limited to GPT-3.5; cross-model validation needed
\item Geometric interpretation remains hypothetical pending attention analysis
\item Semantic classification has subjective elements despite clear criteria
\item Causal relationship between geometry and phases not established
\end{itemize}

\section{Future Work}

Priority research directions include:

\begin{enumerate}
\item \textbf{Direct Validation}: Analyze attention patterns in accessible models (GPT-2, BERT)
\item \textbf{Cross-Model Testing}: Replicate phase analysis across architectures
\item \textbf{Causal Experiments}: Modify attention patterns to test phase changes
\item \textbf{Automated Classification}: Develop robust automated phase detection
\item \textbf{Geometric Priming}: Test if geometric visual priming affects phase distribution
\end{enumerate}

\section{Conclusion}

We documented a statistically significant phase distribution in GPT-3.5 with a consistent 9.7\% transformation bottleneck. This empirical finding reveals a fundamental constraint in current language models' ability to generate genuinely novel insights.

Our proposed geometric interpretation offers a theoretical framework for understanding these patterns, though it remains hypothetical pending validation. The framework generates specific, testable predictions that can be evaluated through attention mechanism analysis.

Whether the constraint is geometric or stems from other architectural factors, the 9.7\% bottleneck represents a key limitation in current AI systems. Understanding and potentially overcoming this constraint could be crucial for developing more creative and transformative AI systems.

The intersection of empirical observation with geometric theory opens new avenues for both interpretability research and architectural innovation in transformer models.

\section*{Acknowledgments}

We thank the open-source community for enabling transparent and reproducible research.

\section*{Code and Data Availability}

All analysis code and data available at: \url{https://github.com/HillaryDanan/multi-geometric-attention}

\end{document}